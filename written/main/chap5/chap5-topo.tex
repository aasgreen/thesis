\documentclass[aagreenthesis]{subfiles}
%\usepackage{macros}
\begin{document}
\chapter{The Applicability of the XY Model to Freely-Suspended Films}
In this chapter, I discuss the applicability of the XY model to systems of
freely-suspended smectic liquid crystals.
\section{Introduction}
As freely-suspended films of liquid crystals are one of the most
two-dimensional systems known, as the crystalline order along the smectic
layer-normal supresses motion along that axis, it is only natural to ask how
well they are described by the classic models of two-dimensions. If the liquid
aspect of the smectic liquid crystal is the focus, this question led to the
development of the Saffman-Delbruck (and later extension to the HPW model)
model, where, though the smectic fluid can indeed be described as a
two-dimensional fluid, it has signifigant corrections that arise from coupling
to the air that surrounds it. This leads to an overall description that is
``quasi-2D'', where the essential three dimensional nature of our universe
cannot be neglected.

Due to the overwhelming success of a hydrodynamic model in capturing the dynamics of
freely-suspended films, it may seem foolish to attempt to describe the same
system of smectic films with a model that completely neglects hydrodynamics, but
that is exactly what I am going to do in this chapter, by studying the
applicability of the XY model to systems of freely-suspended smectic liquid
crystals.
\subsection{The XY Model}
The XY model has a rich history, culminating in a Nobel prize in 2016 to
Professor Thouless, Professor Haldane, and Professor Kosterlitz for "...for
theoretical discoveries of topological phase transitions and topological
phases of matter."

In its simplest realization, the XY model is a two-dimensional grid, with
every point on the grid having a spin (a 2D vector with unit length) associated
with it. Interactions are described by the XY Hamiltonian:
\begin{align}
    H &= -J\sum_{ij} \vec{s}_i\cdot\vec{s}_j
      &= -J\sum_{ij} \cos(\theta_i-\theta_j)
\end{align}
Where $\vec{s}_i$ is the spin-vector at location $i$, which, because it has unit
length and is in 2D can be written as a simple angle $\vec{s}_i =
1\exp(i\theta_i)$ in the complex plane. The sum is over nearest
neighbors.

The XY model is deceptively simple to write down, as it supports a wealth
of emergent properties who defy simple characterization. The first of these
properties that must be discussed are vortices, as they form the backbone
of all interesting topological properties in the XY model.




\subsection{Experimental Realizations of the XY Model in Liquid Crystal Systems}
\section{Quenching Freely-Suspended Films}
\subsection{Past Work}
\subsection{Experimental Design}
\subsection{Early Time Dynamics}

\section{Analyzing Early Time Dynamics of Freely-Suspended Films}
\subsection{Human Annotation}
\subsection{Autocorrelation}
\subsection{Machine Learning Methods}
\section{Results}
\subsection{Dynamic Scaling in Freely-Suspended Films}
\subsection{Logarithmic Corrections in Freely-Suspended Films}
\subsection{Beyond The XY Model}



\end{document}

