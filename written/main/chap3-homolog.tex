\documentclass[aagreenthesis]{subfiles}
\begin{document}
\chapter{Future Work: Structure-Property in the \nfour{} Homologous Series of Molecules}

Now that the existance of the \smcpalpha{} and \smcapa{} phase have been put on
a firm footing, we can extend this study to a broad category of molecules that
share the same core, but have variations in the tail length. Our previous study
focussed on \nfour{}, with a tail length of n=14 carbons. Inspired by
previous studies by the Dublin
group\cite{SreenilayamSpontaneoushelixformation2016,SreenilayamDevelopmentferroelectricitysmectic2017,VijInvestigationheliconicalsmectic2019},
where the n16 and n18 compounds were studied and a helical phase was tentatively
determined based on periodicities of around \SI{5}{\nano\meters} see in atomic
force microscopy images.
this is a test
With the ability to directly probe the periodicity of these compounds with
carbon K-edge resonant X-ray, we have the ability to provide direct evidence for
helical phases in these molecules. The high-resolution provide by RSoXS
additionally allows for a rich study of structure-property relationships, where
the nature of the helical phase can be connected to the tail-length.

Interest in the \nfour{} series of molecules is not confined to the discovery of
helicity. The additional discovery of a bent-core de Vries like phase is also a
motivating interest for study of this series. 

This chapter is organized under discussion of these two phases. First the
helical, smectic \smcpalpha{} is presented and discussed for the n12, n14, and n16 molecules.
Then the bent-core de Vries phase is presented and discussed for this series.

\section{The \smcpalpha{} Phase: Smectic Chirality Beyond the B2 Phase}


\section{The bent-core de Vries phase}

The evidence for a bent-core de Vries phase in \nfour{} will be briefly
reviewed. It is worth noting that the question of ``what is a de Vries phase''
is not settled, even for the simpler phases of rod-shaped molecules, see Jan
Lagerwall's thesis\cite{jansThesis} for an excellent history and discussion of
de Vries phases in the calamitic paradigm. For the purposes of this thesis, we
take the broadest view of a de Vries: it is a tilted phase with no long-range
order present in the tilt order parameter.

Even this broad definition is not free from ambiguity--- what entails long-range
order, and does it matter how the escape from long-range order is achieved? We will also see in this Chapter that the de Vries phase can be
suppressed quite easily by a strong-alignment layer.



To satisfy our definition, we must prove two things about the phase under
investigation to say it is a de Vries bent-core phase. First, it must be tilted.
Second, it must have no long-range order present. Both of these facts are very
difficult to establish in a postive manner, as there is no `smoking-gun'
evidence for either.

We claim that \nfour{} is tilted for the following reasons: the smectic layer spacing is
significantly less than the extended molecular length; the lower temperature
phases are proven to be chiral (therefore tilted), and there is no observed
smectic layer-contraction observed on cooling, which would be expected for an
orthogonal$\rightarrow$ tilted phase transition; and there is no measured change in the birefringence which would be predicted from an orthogonal phase to the
helical \smcpalpha{} (confirmed from RSoXS). Though individually, none of
these facts are conclusive evidence for tilt, all of them in
aggragate suggest
that the highest temperature smectic phase of \nfour{} is tilted.
To see that there is no long-range order in the tilt order parameter, we can
look at the textures of this phase and observe that they look orthogonal-- they
have the clean lines of a focal conic SmA phase. This
is neccesary but not sufficient, as other phase-- such as the \smcapa{} phase --
can also appear to be orthogonal, yet have a definite anti-clinic long range
order. This anti-clinic order (and other, more exotic long helical ordering of
the tilt order paramter) would however, show up in resonant x-ray
scattering. The fact
that the resonant x-ray scattering shown in \autoref{fig:pal30:rsoxs} reveals
zero periodic structure for \nfour{} confirms that the tilt order in the Sm1
phase has no long-range order.

The actual way that this long-range order is escaped is still ambigious. The simplest model,
where the molecules are rotating freely around a cone whose opening angle is
set by the molecular tilt seems unlikely from simple enthalpic space-filling
arguments which would seem to exclude a model where neighboring molecules can be
pointing in completely opposite directions with no energy cost. On the surface,
this seems to be false, but close observation of the c-director flucuations
that occur in freely-suspended-films shows that at the molecular level the
c-director can easily be flucuating with $\pi$-rotations measured against
neighboring molecules, so we cannot rule out this type of behaviour purely on
enthalpic concerns.\todo{need to find de gennes/noels argument for this} This is
fluid-like model, where the molecules are actively rotating with no regard for their
neighbors space.

By including these enthalpic considerations we can build model where short-range order persists, but the size of
these ordered domains is small, and they are randomally distributed such that
an ensemble average of the tilt is zero. This is more of a disordered-crystal
model, where short-range order persists, but is distributed in such a way that the
macroscopic order parameter is still zero-- much like the example of a cooled
ising model developing local ferroelectric ordering of spins, even though the
total magnetization remains zero. 

Regardless of how this long-range tilt order is broken, because these are polar
molecules, we also have to consider the polar-ordering present in any phase.
The way that the polar and tilt interact in the high-temperature de Vries SmA of
\nfour{} is unique to bent-cores.

\begin{figure}[h!]
    \centering
    \includegraphics[width=.6\textwidth]{./figs/pal30/prc/PRCvsTilt/PRCVtilt.png}
    \caption{\label{fig:homog:pal30pvt} The polarization current and the optical
    tilt plotted as a function of applied field strength for both the Sm1
(de Vries SmA) (a,b) and the Sm2 (\smcpalpha{}) (c-d). The polarization current
has additionally been integrated to calculate the time-dependant net
polarization. The threshold electric field ($E_\textrm{th}$) required to manifest the tiger-stripes can
be directly read from the green curve denoting the optical tilt (for
T=\SI{110}{\degreeCelsius}, $E_\textrm{th} \approx \SI{5}{\volt\per\micro\metre}$),
and the saturation electric field where the net polarization is no longer
changing ($E_\textrm{sat}$) can be directly
read from the red curve, which denotes the net polarization, (for
T=\SI{110}{\degreeCelsius}, $E_\textrm{sat} \approx \SI{5}{\volt\per\micro\metre}$).
Both $E_\textrm{th}$ and $E_\textrm{sat}$ are plotted in as the inset of
\autoref{fig:threshold}. }
\end{figure}

The polarization current has been integrated to give the net polarization, which
is a direct measure of the ensemble average of the molecular polar orientation with an
applied electric field. The optical tilt measurements were done by analyzing the
contrast between adjacent stripes\todo{small figure showing this}. In the ground
state, where there is no long-range order present in tilt, the contrast is zero.
At a field whose strength is over some threshold value ($E_\textrm{th}$), the
state transitions into a tilted (therefore chiral) state. The handedness of the
domain sets the tilt direction. This tilt can be directly measured as it is
proportional to the contrast between different handed domains:
\begin{equation}
    \textrm{contrast}\propto \sin(4\theta)
\end{equation}\todo{check this}.
Barring pathological examples, \textit{any} alignment, orientation, or movement
of the molecular long axis will show up as the observable seperation of the
texture into chiral domains. 


With some unique exceptions\cite{michi}\todo{cite michi}
most bent-core systems have the tilt and
polarity strongly coupled: where one moves, the other follows. Through symmetry,
this can be expressed as the condition that rotations around the molecular long
axis are forbidden: to transition to different states, the molecule must rotate
around the smectic layer normal, confined to the tilt-cone.
\autoref{fig:homog:pal30pvt} clearly shows that for the Sm1 phase of
\nfour{}, this is no longer the case. The movement of the polar
director is completely decoupled from the
movement of the c-director, which can only occur if the \nfour{} molecules
are rotating around the long axis, allowing their polarization to change while
the optical tilt stays fixed.\todo{small inline figure of this} It is only
after the polarization saturates and is completely aligned
before there is any detectable movement of the tilt-director. The previous
examples of this occuring in bent-cores required pulse-engineering the applied
voltage,
where a very large, fast voltage needed to be applied to see rotations around
the long axis of the molecule. \nfour{} is the first molecule where the polar
and tilt order are decoupled enough to allow these long-axis rotations
naturally.

This decoupling results in 
an important distinction because, unlike their calamitic cousins, these
bent-core molecules are not inherently chiral. They are only develop chirality
once they pack into phases. they require tilt and smectic ordering
before the mirror-symmetry breaking required for chirality is achieved. They
also require that the tilt and polar order are strongly coupled--free rotations
around the long-axis result in a return of mirror-symmetry and the chirality of
the phase is lost. In contrast to the calamtics, who merely have to `discover'
their own inherent chirality once a field is applied (for instance, in the
electroclinic effect), the bent-cores must spontaneously create their own. This
is not a new discovery, the B2 phases\cite{link_spontaneous_1997} have long been
known to develop spontaneous chirality on cooling from an achiral phase. This
is however
the first discovery of field-activated chirality in a bent-core system, where
chirality is induced at a certain voltage, but then returns to an achiral ground
state when the voltage is removed.

The bent-core de Vries phase we discovered in \nfour{} has three defining
characteristics: it is tilted; it has no long-range tilt-order present, and the
polarity and tilt are decoupled.

One potential approach to modelling this phase is the Landau-de Gennes
model put forth by Eremin et al.\cite{eremin2008electrically} where they
discovered an electro-clinic analog in a hockey-stick compound. 
\todo{get and show the molecule}
\begin{figure}
    \centering
    \includegraphics[width=\textwidth]{./figs/pal30/fromPapers/tiltVfield.jpg}
    \caption{\label{fig:achiralTilt} Experimental data for `hockey-stick'
    compound studied by Eremin et al.\cite{eremin2008electrically} Note, the threshold for optical tilt. The lines guide the only.}
\end{figure}
Broadly, their model describes an orthogonal bent-core phase where the polar
director is free to move. On application of an electric-field, the polarization
of the molecules orients. An enthalpic\todo{check this with ed} phase transition
is then driven by excluded-volume interactions, where the molecules can pack
more efficiently by tilting. 

This interaction is described by the following Landau-de Gennes free energy:

\begin{multline}
    f(P,\theta,T,E) = \overbrace{\left(a_0(T-T_\theta)\theta^2+b\theta^4+c\theta^6
    \right)}^{f_\theta}\\
    + \underbrace{\left( \alpha_0(T-T_P)P^2+\beta P^4 + \gamma P^6 -PE
            \right)}_{f_P} +\underbrace{\left(
    -\Gamma (P \theta)^2 \right)}_{f_{\theta P}}
\end{multline}

Though this model was originally formulated to describe the transition from an
orthogonal phase to a tilted one, we can adapt it to our bent-core de Vries
phase by changing the interpretation of $\theta$ from the molecular tilt to the
optical tilt.

Though this model succesfully captures the overall nature of the
electically-driven onset of chirality in the bent-core de Vries phase, where the
molecules first orient their polar directors to the field, and on the
achievement of total polar alignment, an optical tilt develops, it leaves much
to be desired. Because this model does not demand details of the
microscopic interactions that lead to the observed transition, there is an abundance of
fitting parameters that will lead to overfitting and the model loses most of its
predictive power.

Because of this, efforts must be undertaken to develop a microscopic model of
this de Vries phase. 


\subsubsection{The Beginnings of a Microscopic Model for the Bent-Core de Vries}

Though the full development of this theory is beyond the
scope of this thesis, I will outline the steps that must be taken.

First, the decoupling present between the polar and tilt order in the bent-core
de Vries phase has to be experimentally
quantified. This decoupling is realized when the molecule can rotate freely
around the long axis, which can be described by a single angle, $\beta$. The other switching mode, common to convential bent-cores,
is described by a rotation around the tilt cone, described by an angle $\theta$.
\todo{include math section showing the rotation of this angle}
\todo{using michi's model, discuss how we can quantify which mode of switching
we are in}



\subsection{Textures of bent-code de Vries phase}


\begin{figure}[h!]
    \centering
    \includegraphics[width=.8\textwidth]{figs/pal30/textureSM2/sm1Textures100.png}
    \caption{\label{}}
\end{figure}


\subsection{Electro-optics of bent-core de Vries phase}
text
\begin{figure}[h!]
    \centering
    \includegraphics[width=.8\textwidth]{figs/pal30/prc/spacedSm1PRC.png}
    \caption{\label{}}
\end{figure}
text

\begin{figure}[h!]
    \centering
    \includegraphics[width=.8\textwidth]{figs/pal30/prc/3dplot-sm1.png}
    \caption{\label{}}
\end{figure}
text

\subsection{X-ray analysis of bent-core de Vries phase}

\begin{figure}[h!]
    \centering
    \includegraphics[width=.8\textwidth]{figs/pal30/xraysm1/rsosxSmaT113-modified.png}
    \caption{\label{}}
\end{figure}


\begin{figure}[h!]
    \centering
    \includegraphics{figs/pal30/xraysm1/sm1-saxs-annote.png}
    \caption{\label{}}
\end{figure}




\section{Discussion for the bent-core de Vries phase}
\begin{figure}[h!]
    \centering
    \includegraphics{figs/pal30/deVries/dvAlign.png}
    \caption{\label{}}
\end{figure}

\section{Alignment-Induced Phase Transitions}
It is easy to ignore that when experimentalists preform phase characterization
of liquid crystals, we are not only characterizing the phase in terms of the
usual statistical mechanic variables of pressure and temperature, we also must
include the variable of how this material was confined. An example of this is
the suppression of the long-range helix in the SmC* anti-ferro phase\todo{cite
noel's flc stuff} in thin cells. And for the majority of liquid crystal/soft
matter systems, the conditions of confinement do not impact the bulk phase that
dramatically, and we are right to ignore it.

However, our investigation of the homologous series of molecules related to
\nfour{} have revealed that the confinement conditions cannot be ignored for
this series of molecules. \autoref{fig:n16textureCompare} shows the same
material, \nsix{} in two different cells. One is an LC Vision cell, one is an
Instec cell.
\todo{need to find out what is used for the alignment layer}.

The behaviour of the same material in these different cells is so different as
to be a different phase. This observation goes a long way to explain why the
original phase classification of \nfour{} was so contensious- it could be that
the Dublin group was actually observing a different phase from our observations
in Boulder.

Though we are currently limited in our ability to compare the behaviour of these
materials in different cells due to a dearth of available material, we can
compare the electro-optic behaviour at different temperatures.

\nsix{} begins switching at similiar temperatures on cooling from the isotropic
phase, shown in \autoref{fig:future:n16t116}. In LC vision cells (which were used to characterize the \nfour{}
compound), the response to an applied electric field manifests as the familiar
tiger strips seen in \nfour{} (\autoref{fig:future:n16t116} (a-c)), with a subtle difference: while the tiger stripes
seen in \nfour{} were parallel to the smectic layers \textit{on average}, they
were wavy. The tiger-stripes seen in \nsix{} are straight, indicating that they
are rigidly bound to the smectic layers. The implications of this are not
understood.

In the Instec cells, the response to an applied electric field manifest as a
barely distinguishable change in contast in large block-like regions
(\autoref{fig:future:n16t116}(d-f)). However,
these phases look broadly similar-- an orthogonal-looking ground state that
switches into some chiral phase. It is only on further cooling that the texture
undergo a dramatic divergence.


\begin{figure}[h!]
    \centering
    \includegraphics[width=\textwidth]{figs/pal30/future/LCvIn116.png}
    \caption{\label{fig:future:n16t116} Planar textures of \nsix{} at approximately
        \SI{116}{\degreeCelsius} in an LC vision cell
    (a-c) and an Instec cell (d-f) viewed under cross polarizers. Both cells use a polyimide anti-parallel
alignment layer. The LC vision cell has clear focal conics, while the Instec
cell is entirely aligned. Both textures manifest chirality, seen as the distinct
difference between the textures with a negative or positive out-of-plane
electric field applied. The contrast between handed domains shows up clearly in
the LC vision cells (compare (g) and (i)) while the contrast in the Instec cells
is barely distinguishable. }
\end{figure}


Cooling to \SI{110}{\degreeCelsius}  gives a familiar \smcapa{} like texture in
the LC vision cells that looks identical to \nfour{}-- an orthogonal like phase
that switches into a clearly chiral phase, with contrast between domains of
alternating chirality (\autoref{fig:future:n16t110} (a-c)). However, the Instec cell appears to not be chiral-- the
positive and negative applied field result in a texture that appears broadly
identical, indicating a mirror symmetry in the phase that precludes
chirality (\autoref{fig:future:n16t110} (d-f)). Instead, the birefringence
increases in broad stripes that are parallel to the smectic layers. One possible
explanation is that the alignment of the bottom-plate and top-plate of the cell
aligned at opposite positions on the tilt-cone. This would give a texture that
could appear orthogonal, even when the polarization has been universally aligned
to an external field. This is not unheard of, and one way to rule this out would
be to prepare a cell with an alignment layer on only one of the plates of the
cell. The other possible explanation is that the dramatic alignment caused by
the Instec cell, alongwith the unique decoupling of polarity and tilt seen in
these molecules, are interacting in an unforeseen way. 

\begin{figure}[h!]
    \centering
    \includegraphics[width=\textwidth]{figs/pal30/future/LCvIn110.png}
    \caption{\label{fig:future:n16t110}  Planar textures of \nsix{} at
        approximately \SI{110}{\degreeCelsius} in an LC vision cell
    (a-c) and an Instec cell (d-f) viewed under cross polarizers. Both cells use a polyimide anti-parallel
alignment layer. The LC vision cell has clear focal conics, while the Instec
cell is entirely aligned. Both textures manifest chirality, seen as the distinct
difference between the textures with a negative or positive out-of-plane
electric field applied. The contrast between handed domains shows up clearly in
the LC vision cells (compare (g) and (i)) while the contrast in the Instec cells
is barely distinguishable. }
\end{figure}

Further cooling to \SI{100}{\degreeCelsius} gives a unique texture in both the
LC Vision cell, and the Instec cell, see \autoref{fig:future:n16t100}. While the \nfour{} textures in this phase
had regular stiations of contrast present when an external field was applied,
they did not manifest the riot of colour present in the \nsix{} textures, where
bands of different birefringences appear overlaid bands of
contrast\autoref{fig:future:n16t100}(a-c). The cause
of this is currently unknown. It is also worth noting that this state appears to
be long-lived, as inspection of the texture that was previously exposed to an
electric field in \autoref{fig:future:n16t100} (b), still manifest faint
striations, indicating that it has not fully relaxed to an achiral ground-state.  

This stands in contrast to the Instec cell, whose response under an applied
electric field has actually decreases. The bands of birefringence clearly
visible at \SI{110}{\degreeCelsius} in \autoref{fig:future:n16t110} (d,f) have
faded. However, the dark bands that form orthogonal to the layers (much like
those caused by
undulating layers in a SmA focal conic) which were only faintly visible at
\SI{110}{\degreeCelsius} in \autoref{fig:future:n16t110} (d,f) are now more
pronounced.

\begin{figure}[h!]
    \centering
\includegraphics[width=\textwidth]{figs/pal30/future/LCvIn100.png}
    \caption{\label{fig:future:n161t100}  Planar textures of \nsix{} at
        approximately \SI{100}{\degreeCelsius} in an LC vision cell
    (a-c) and an Instec cell (d-f) viewed under cross polarizers. Both cells use a polyimide anti-parallel
alignment layer. The LC vision cell has clear focal conics, while the Instec
cell is entirely aligned. Both textures manifest chirality, seen as the distinct
difference between the textures with a negative or positive out-of-plane
electric field applied. The contrast between handed domains shows up clearly in
the LC vision cells (compare (g) and (i)) while the contrast in the Instec cells
is barely distinguishable. }
\end{figure}

On further cooling, the textures of both cells converge to familiar \smcapf{}
switching, where the entire cell has an increase in birefringence under an
applied electric field. 

Solving this mystery will go a long way to advancing our understanding of how
chirality develops and manifest in these soft-matter systems. We have a phase
with a known chiral bulk-state (see the resonant x-ray scattering of \nsix{} at
\SI{110}{\degreeCelsius}) that, in some cells (LC vision), shows its chiral nature, but in
others (Instec), this chirality seems to be suppressed. The more conventional
explanation, that this is still the same chiral \smcpalpha{} phase which appears
to be achiral because of some fluke in the alignment would need to first be
ruled out. 





\end{document}


